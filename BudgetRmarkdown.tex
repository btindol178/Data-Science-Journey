% Options for packages loaded elsewhere
\PassOptionsToPackage{unicode}{hyperref}
\PassOptionsToPackage{hyphens}{url}
%
\documentclass[
]{article}
\usepackage{lmodern}
\usepackage{amssymb,amsmath}
\usepackage{ifxetex,ifluatex}
\ifnum 0\ifxetex 1\fi\ifluatex 1\fi=0 % if pdftex
  \usepackage[T1]{fontenc}
  \usepackage[utf8]{inputenc}
  \usepackage{textcomp} % provide euro and other symbols
\else % if luatex or xetex
  \usepackage{unicode-math}
  \defaultfontfeatures{Scale=MatchLowercase}
  \defaultfontfeatures[\rmfamily]{Ligatures=TeX,Scale=1}
\fi
% Use upquote if available, for straight quotes in verbatim environments
\IfFileExists{upquote.sty}{\usepackage{upquote}}{}
\IfFileExists{microtype.sty}{% use microtype if available
  \usepackage[]{microtype}
  \UseMicrotypeSet[protrusion]{basicmath} % disable protrusion for tt fonts
}{}
\makeatletter
\@ifundefined{KOMAClassName}{% if non-KOMA class
  \IfFileExists{parskip.sty}{%
    \usepackage{parskip}
  }{% else
    \setlength{\parindent}{0pt}
    \setlength{\parskip}{6pt plus 2pt minus 1pt}}
}{% if KOMA class
  \KOMAoptions{parskip=half}}
\makeatother
\usepackage{xcolor}
\IfFileExists{xurl.sty}{\usepackage{xurl}}{} % add URL line breaks if available
\IfFileExists{bookmark.sty}{\usepackage{bookmark}}{\usepackage{hyperref}}
\hypersetup{
  pdftitle={Budget},
  hidelinks,
  pdfcreator={LaTeX via pandoc}}
\urlstyle{same} % disable monospaced font for URLs
\usepackage[margin=1in]{geometry}
\usepackage{color}
\usepackage{fancyvrb}
\newcommand{\VerbBar}{|}
\newcommand{\VERB}{\Verb[commandchars=\\\{\}]}
\DefineVerbatimEnvironment{Highlighting}{Verbatim}{commandchars=\\\{\}}
% Add ',fontsize=\small' for more characters per line
\usepackage{framed}
\definecolor{shadecolor}{RGB}{248,248,248}
\newenvironment{Shaded}{\begin{snugshade}}{\end{snugshade}}
\newcommand{\AlertTok}[1]{\textcolor[rgb]{0.94,0.16,0.16}{#1}}
\newcommand{\AnnotationTok}[1]{\textcolor[rgb]{0.56,0.35,0.01}{\textbf{\textit{#1}}}}
\newcommand{\AttributeTok}[1]{\textcolor[rgb]{0.77,0.63,0.00}{#1}}
\newcommand{\BaseNTok}[1]{\textcolor[rgb]{0.00,0.00,0.81}{#1}}
\newcommand{\BuiltInTok}[1]{#1}
\newcommand{\CharTok}[1]{\textcolor[rgb]{0.31,0.60,0.02}{#1}}
\newcommand{\CommentTok}[1]{\textcolor[rgb]{0.56,0.35,0.01}{\textit{#1}}}
\newcommand{\CommentVarTok}[1]{\textcolor[rgb]{0.56,0.35,0.01}{\textbf{\textit{#1}}}}
\newcommand{\ConstantTok}[1]{\textcolor[rgb]{0.00,0.00,0.00}{#1}}
\newcommand{\ControlFlowTok}[1]{\textcolor[rgb]{0.13,0.29,0.53}{\textbf{#1}}}
\newcommand{\DataTypeTok}[1]{\textcolor[rgb]{0.13,0.29,0.53}{#1}}
\newcommand{\DecValTok}[1]{\textcolor[rgb]{0.00,0.00,0.81}{#1}}
\newcommand{\DocumentationTok}[1]{\textcolor[rgb]{0.56,0.35,0.01}{\textbf{\textit{#1}}}}
\newcommand{\ErrorTok}[1]{\textcolor[rgb]{0.64,0.00,0.00}{\textbf{#1}}}
\newcommand{\ExtensionTok}[1]{#1}
\newcommand{\FloatTok}[1]{\textcolor[rgb]{0.00,0.00,0.81}{#1}}
\newcommand{\FunctionTok}[1]{\textcolor[rgb]{0.00,0.00,0.00}{#1}}
\newcommand{\ImportTok}[1]{#1}
\newcommand{\InformationTok}[1]{\textcolor[rgb]{0.56,0.35,0.01}{\textbf{\textit{#1}}}}
\newcommand{\KeywordTok}[1]{\textcolor[rgb]{0.13,0.29,0.53}{\textbf{#1}}}
\newcommand{\NormalTok}[1]{#1}
\newcommand{\OperatorTok}[1]{\textcolor[rgb]{0.81,0.36,0.00}{\textbf{#1}}}
\newcommand{\OtherTok}[1]{\textcolor[rgb]{0.56,0.35,0.01}{#1}}
\newcommand{\PreprocessorTok}[1]{\textcolor[rgb]{0.56,0.35,0.01}{\textit{#1}}}
\newcommand{\RegionMarkerTok}[1]{#1}
\newcommand{\SpecialCharTok}[1]{\textcolor[rgb]{0.00,0.00,0.00}{#1}}
\newcommand{\SpecialStringTok}[1]{\textcolor[rgb]{0.31,0.60,0.02}{#1}}
\newcommand{\StringTok}[1]{\textcolor[rgb]{0.31,0.60,0.02}{#1}}
\newcommand{\VariableTok}[1]{\textcolor[rgb]{0.00,0.00,0.00}{#1}}
\newcommand{\VerbatimStringTok}[1]{\textcolor[rgb]{0.31,0.60,0.02}{#1}}
\newcommand{\WarningTok}[1]{\textcolor[rgb]{0.56,0.35,0.01}{\textbf{\textit{#1}}}}
\usepackage{graphicx,grffile}
\makeatletter
\def\maxwidth{\ifdim\Gin@nat@width>\linewidth\linewidth\else\Gin@nat@width\fi}
\def\maxheight{\ifdim\Gin@nat@height>\textheight\textheight\else\Gin@nat@height\fi}
\makeatother
% Scale images if necessary, so that they will not overflow the page
% margins by default, and it is still possible to overwrite the defaults
% using explicit options in \includegraphics[width, height, ...]{}
\setkeys{Gin}{width=\maxwidth,height=\maxheight,keepaspectratio}
% Set default figure placement to htbp
\makeatletter
\def\fps@figure{htbp}
\makeatother
\setlength{\emergencystretch}{3em} % prevent overfull lines
\providecommand{\tightlist}{%
  \setlength{\itemsep}{0pt}\setlength{\parskip}{0pt}}
\setcounter{secnumdepth}{-\maxdimen} % remove section numbering

\title{Budget}
\author{}
\date{\vspace{-2.5em}}

\begin{document}
\maketitle

\hypertarget{budget-analysis-of-monthly-spending}{%
\subsection{Budget Analysis of Monthly
Spending}\label{budget-analysis-of-monthly-spending}}

This is an analysis of monthly speding since the beginning of the year.
The purpose of this document is to understand the average monthly
spending by category and reduce the spending for each category by 10\%
to save money for a house down deposit.

\begin{Shaded}
\begin{Highlighting}[]
\KeywordTok{require}\NormalTok{(dplyr)}
\end{Highlighting}
\end{Shaded}

\begin{verbatim}
## Loading required package: dplyr
\end{verbatim}

\begin{verbatim}
## 
## Attaching package: 'dplyr'
\end{verbatim}

\begin{verbatim}
## The following objects are masked from 'package:stats':
## 
##     filter, lag
\end{verbatim}

\begin{verbatim}
## The following objects are masked from 'package:base':
## 
##     intersect, setdiff, setequal, union
\end{verbatim}

\begin{Shaded}
\begin{Highlighting}[]
\KeywordTok{require}\NormalTok{(tidyverse)}
\end{Highlighting}
\end{Shaded}

\begin{verbatim}
## Loading required package: tidyverse
\end{verbatim}

\begin{verbatim}
## -- Attaching packages ------------------------------------- tidyverse 1.3.0 --
\end{verbatim}

\begin{verbatim}
## v ggplot2 3.3.0     v purrr   0.3.4
## v tibble  3.0.1     v stringr 1.4.0
## v tidyr   1.1.0     v forcats 0.5.0
## v readr   1.3.1
\end{verbatim}

\begin{verbatim}
## -- Conflicts ---------------------------------------- tidyverse_conflicts() --
## x dplyr::filter() masks stats::filter()
## x dplyr::lag()    masks stats::lag()
\end{verbatim}

\begin{Shaded}
\begin{Highlighting}[]
\KeywordTok{require}\NormalTok{(lubridate)}
\end{Highlighting}
\end{Shaded}

\begin{verbatim}
## Loading required package: lubridate
\end{verbatim}

\begin{verbatim}
## 
## Attaching package: 'lubridate'
\end{verbatim}

\begin{verbatim}
## The following objects are masked from 'package:dplyr':
## 
##     intersect, setdiff, union
\end{verbatim}

\begin{verbatim}
## The following objects are masked from 'package:base':
## 
##     date, intersect, setdiff, union
\end{verbatim}

\begin{Shaded}
\begin{Highlighting}[]
\KeywordTok{require}\NormalTok{(ggplot2)}
\KeywordTok{require}\NormalTok{(cowplot)}
\end{Highlighting}
\end{Shaded}

\begin{verbatim}
## Loading required package: cowplot
\end{verbatim}

\begin{verbatim}
## 
## ********************************************************
\end{verbatim}

\begin{verbatim}
## Note: As of version 1.0.0, cowplot does not change the
\end{verbatim}

\begin{verbatim}
##   default ggplot2 theme anymore. To recover the previous
\end{verbatim}

\begin{verbatim}
##   behavior, execute:
##   theme_set(theme_cowplot())
\end{verbatim}

\begin{verbatim}
## ********************************************************
\end{verbatim}

\begin{verbatim}
## 
## Attaching package: 'cowplot'
\end{verbatim}

\begin{verbatim}
## The following object is masked from 'package:lubridate':
## 
##     stamp
\end{verbatim}

\hypertarget{importing-the-budget-dataset}{%
\subsection{Importing the budget
dataset}\label{importing-the-budget-dataset}}

First we upload budget then make a month column

\begin{verbatim}
##        date                  item            category  cost month
## 1  6/2/2020    MCDONALD'S F13014   Restaurants/Dining  4.12     6
## 2  6/2/2020       MEDIUM MONTHLY          Advertising  5.00     6
## 3  6/2/2020 WAL-MART SUPERCENTER  General Merchandise 31.99     6
## 4  6/1/2020           ALDI 67041            Groceries 18.53     6
## 5  6/1/2020     MCDONALD'S F2311   Restaurants/Dining  3.59     6
## 6 5/31/2020            WEST MAIN        Entertainment 12.70     5
\end{verbatim}

\hypertarget{aggregate-by-category-to-get-total-sum-of-cost}{%
\subsection{Aggregate by category to get total sum of
cost}\label{aggregate-by-category-to-get-total-sum-of-cost}}

Aggregate by category and month to get sum of category by month

\begin{Shaded}
\begin{Highlighting}[]
\NormalTok{budget_f <-}\StringTok{ }\KeywordTok{aggregate}\NormalTok{(budget[,}\KeywordTok{c}\NormalTok{(}\DecValTok{4}\NormalTok{)], }\DataTypeTok{by =} \KeywordTok{list}\NormalTok{(budget}\OperatorTok{$}\NormalTok{category,budget}\OperatorTok{$}\NormalTok{month), }\DataTypeTok{FUN=}\NormalTok{ sum)}
\KeywordTok{head}\NormalTok{(budget_f)}
\end{Highlighting}
\end{Shaded}

\begin{verbatim}
##                    Group.1 Group.2       x
## 1 Cable/Satellite Services       1   60.00
## 2     Credit Card Payments       1 4196.90
## 3   Dues and Subscriptions       1    6.88
## 4                Education       1   15.85
## 5              Electronics       1   48.00
## 6            Entertainment       1  114.34
\end{verbatim}

\hypertarget{rename-columns-order-the-dataframe-by-month-and-cost}{%
\subsection{Rename columns order the dataframe by month and
cost}\label{rename-columns-order-the-dataframe-by-month-and-cost}}

Manipulate dataframe to have proper column format as numeric and change
column names

\begin{Shaded}
\begin{Highlighting}[]
\KeywordTok{colnames}\NormalTok{(budget_f)[}\DecValTok{1}\NormalTok{] <-}\StringTok{ "category"}
\KeywordTok{colnames}\NormalTok{(budget_f)[}\DecValTok{2}\NormalTok{] <-}\StringTok{ "month"}
\KeywordTok{colnames}\NormalTok{(budget_f)[}\DecValTok{3}\NormalTok{] <-}\StringTok{ "cost"}
\NormalTok{budget_f <-}\StringTok{ }\NormalTok{budget_f[}\KeywordTok{order}\NormalTok{(budget_f}\OperatorTok{$}\NormalTok{month, budget_f}\OperatorTok{$}\NormalTok{cost),]}
\NormalTok{budget_f}\OperatorTok{$}\NormalTok{cost <-}\StringTok{ }\KeywordTok{as.numeric}\NormalTok{(budget_f}\OperatorTok{$}\NormalTok{cost)}
\NormalTok{budget_f <-}\StringTok{ }\KeywordTok{as.data.frame}\NormalTok{(budget_f)}
\NormalTok{budget_f <-}\StringTok{ }\NormalTok{budget_f[}\OperatorTok{!}\NormalTok{budget_f}\OperatorTok{$}\NormalTok{category }\OperatorTok{==}\StringTok{ "Credit Card Payments"}\NormalTok{,]}

\KeywordTok{head}\NormalTok{(budget_f)}
\end{Highlighting}
\end{Shaded}

\begin{verbatim}
##                  category month  cost
## 3  Dues and Subscriptions     1  6.88
## 14   Service Charges/Fees     1 13.45
## 4               Education     1 15.85
## 12          Personal Care     1 18.70
## 10                Hobbies     1 18.87
## 5             Electronics     1 48.00
\end{verbatim}

\hypertarget{use-ggplot2-to-visualize}{%
\section{Use ggplot2 to visualize}\label{use-ggplot2-to-visualize}}

\begin{Shaded}
\begin{Highlighting}[]
\NormalTok{plot_base2 <-}\StringTok{ }\KeywordTok{ggplot}\NormalTok{(}\DataTypeTok{data =}\NormalTok{ budget_f, }\DataTypeTok{mapping =} \KeywordTok{aes}\NormalTok{(}\DataTypeTok{x =} \KeywordTok{reorder}\NormalTok{(category, }\OperatorTok{-}\NormalTok{cost), }\DataTypeTok{y =}\NormalTok{ cost))}

\CommentTok{# save a better-formatted version of the base plot in "plot_base_clean"}
\NormalTok{plot_base_clean2 <-}\StringTok{ }\NormalTok{plot_base2 }\OperatorTok{+}\StringTok{ }
\StringTok{  }\CommentTok{# apply basic black and white theme - this theme removes the background colour by default}
\StringTok{  }\KeywordTok{theme_bw}\NormalTok{() }\OperatorTok{+}\StringTok{ }
\StringTok{  }\CommentTok{# remove gridlines. panel.grid.major is for vertical lines, panel.grid.minor is for horizontal lines}
\StringTok{  }\KeywordTok{theme}\NormalTok{(}\DataTypeTok{panel.grid.major =} \KeywordTok{element_blank}\NormalTok{(), }\DataTypeTok{panel.grid.minor =} \KeywordTok{element_blank}\NormalTok{(),}
        \CommentTok{# remove borders}
        \DataTypeTok{panel.border =} \KeywordTok{element_blank}\NormalTok{(),}
        \CommentTok{# removing borders also removes x and y axes. Add them back}
        \DataTypeTok{axis.line =} \KeywordTok{element_line}\NormalTok{())}

\CommentTok{# Plot stuff}
\NormalTok{plot1f <-}\StringTok{ }\NormalTok{plot_base_clean2 }\OperatorTok{+}\StringTok{ }\KeywordTok{geom_bar}\NormalTok{(}\DataTypeTok{data =} \KeywordTok{subset}\NormalTok{(budget_f, month }\OperatorTok{==}\StringTok{ }\DecValTok{1}\NormalTok{), }\DataTypeTok{stat =} \StringTok{"identity"}\NormalTok{, }\KeywordTok{aes}\NormalTok{(}\DataTypeTok{fill =}\NormalTok{ category))  }
\NormalTok{plot2f <-}\StringTok{ }\NormalTok{plot_base_clean2 }\OperatorTok{+}\StringTok{ }\KeywordTok{geom_bar}\NormalTok{(}\DataTypeTok{data =} \KeywordTok{subset}\NormalTok{(budget_f, month }\OperatorTok{==}\StringTok{ }\DecValTok{2}\NormalTok{), }\DataTypeTok{stat =} \StringTok{"identity"}\NormalTok{, }\KeywordTok{aes}\NormalTok{(}\DataTypeTok{fill =}\NormalTok{ category))}
\NormalTok{plot3f <-}\StringTok{ }\NormalTok{plot_base_clean2 }\OperatorTok{+}\StringTok{ }\KeywordTok{geom_bar}\NormalTok{(}\DataTypeTok{data =} \KeywordTok{subset}\NormalTok{(budget_f, month }\OperatorTok{==}\StringTok{ }\DecValTok{3}\NormalTok{), }\DataTypeTok{stat =} \StringTok{"identity"}\NormalTok{, }\KeywordTok{aes}\NormalTok{(}\DataTypeTok{fill =}\NormalTok{ category))}
\NormalTok{plot4f <-}\StringTok{ }\NormalTok{plot_base_clean2 }\OperatorTok{+}\StringTok{ }\KeywordTok{geom_bar}\NormalTok{(}\DataTypeTok{data =} \KeywordTok{subset}\NormalTok{(budget_f, month }\OperatorTok{==}\StringTok{ }\DecValTok{4}\NormalTok{), }\DataTypeTok{stat =} \StringTok{"identity"}\NormalTok{, }\KeywordTok{aes}\NormalTok{(}\DataTypeTok{fill =}\NormalTok{ category))}
\NormalTok{plot5f <-plot_base_clean2 }\OperatorTok{+}\StringTok{ }\KeywordTok{geom_bar}\NormalTok{(}\DataTypeTok{data =} \KeywordTok{subset}\NormalTok{(budget_f, month }\OperatorTok{==}\StringTok{ }\DecValTok{5}\NormalTok{), }\DataTypeTok{stat =} \StringTok{"identity"}\NormalTok{, }\KeywordTok{aes}\NormalTok{(}\DataTypeTok{fill =}\NormalTok{ category))}
\NormalTok{plot6f <-}\StringTok{ }\NormalTok{plot_base_clean2 }\OperatorTok{+}\StringTok{ }\KeywordTok{geom_bar}\NormalTok{(}\DataTypeTok{data =} \KeywordTok{subset}\NormalTok{(budget_f, month }\OperatorTok{==}\StringTok{ }\DecValTok{6}\NormalTok{), }\DataTypeTok{stat =} \StringTok{"identity"}\NormalTok{, }\KeywordTok{aes}\NormalTok{(}\DataTypeTok{fill =}\NormalTok{ category))}
\NormalTok{plot1f <-}\StringTok{ }\NormalTok{plot1f}\OperatorTok{+}\StringTok{ }\KeywordTok{labs}\NormalTok{(}\DataTypeTok{title=} \StringTok{"Sum of spending by categories January"}\NormalTok{, }\DataTypeTok{y=}\StringTok{"Spent"}\NormalTok{, }\DataTypeTok{x =} \StringTok{"Category"}\NormalTok{) }\OperatorTok{+}\KeywordTok{theme}\NormalTok{(}\DataTypeTok{text =} \KeywordTok{element_text}\NormalTok{(}\DataTypeTok{size=}\DecValTok{10}\NormalTok{), }\DataTypeTok{axis.text.x =} \KeywordTok{element_text}\NormalTok{(}\DataTypeTok{angle=}\DecValTok{90}\NormalTok{, }\DataTypeTok{hjust=}\DecValTok{1}\NormalTok{)) }
\NormalTok{plot2f <-}\StringTok{ }\NormalTok{plot2f}\OperatorTok{+}\StringTok{ }\KeywordTok{labs}\NormalTok{(}\DataTypeTok{title=} \StringTok{"Sum of spending by categories Febuary"}\NormalTok{, }\DataTypeTok{y=}\StringTok{"Spent"}\NormalTok{, }\DataTypeTok{x =} \StringTok{"Category"}\NormalTok{) }\OperatorTok{+}\KeywordTok{theme}\NormalTok{(}\DataTypeTok{text =} \KeywordTok{element_text}\NormalTok{(}\DataTypeTok{size=}\DecValTok{10}\NormalTok{), }\DataTypeTok{axis.text.x =} \KeywordTok{element_text}\NormalTok{(}\DataTypeTok{angle=}\DecValTok{90}\NormalTok{, }\DataTypeTok{hjust=}\DecValTok{1}\NormalTok{)) }
\NormalTok{plot3f <-}\StringTok{ }\NormalTok{plot3f}\OperatorTok{+}\StringTok{ }\KeywordTok{labs}\NormalTok{(}\DataTypeTok{title=} \StringTok{"Sum of spending by categories March"}\NormalTok{, }\DataTypeTok{y=}\StringTok{"Spent"}\NormalTok{, }\DataTypeTok{x =} \StringTok{"Category"}\NormalTok{) }\OperatorTok{+}\KeywordTok{theme}\NormalTok{(}\DataTypeTok{text =} \KeywordTok{element_text}\NormalTok{(}\DataTypeTok{size=}\DecValTok{10}\NormalTok{), }\DataTypeTok{axis.text.x =} \KeywordTok{element_text}\NormalTok{(}\DataTypeTok{angle=}\DecValTok{90}\NormalTok{, }\DataTypeTok{hjust=}\DecValTok{1}\NormalTok{)) }
\NormalTok{plot4f <-}\StringTok{ }\NormalTok{plot4f}\OperatorTok{+}\StringTok{ }\KeywordTok{labs}\NormalTok{(}\DataTypeTok{title=} \StringTok{"Sum of spending by categories April"}\NormalTok{, }\DataTypeTok{y=}\StringTok{"Spent"}\NormalTok{, }\DataTypeTok{x =} \StringTok{"Category"}\NormalTok{) }\OperatorTok{+}\KeywordTok{theme}\NormalTok{(}\DataTypeTok{text =} \KeywordTok{element_text}\NormalTok{(}\DataTypeTok{size=}\DecValTok{10}\NormalTok{), }\DataTypeTok{axis.text.x =} \KeywordTok{element_text}\NormalTok{(}\DataTypeTok{angle=}\DecValTok{90}\NormalTok{, }\DataTypeTok{hjust=}\DecValTok{1}\NormalTok{)) }
\NormalTok{plot5f <-}\StringTok{ }\NormalTok{plot5f}\OperatorTok{+}\StringTok{ }\KeywordTok{labs}\NormalTok{(}\DataTypeTok{title=} \StringTok{"Sum of spending by categories May"}\NormalTok{, }\DataTypeTok{y=}\StringTok{"Spent"}\NormalTok{, }\DataTypeTok{x =} \StringTok{"Category"}\NormalTok{) }\OperatorTok{+}\KeywordTok{theme}\NormalTok{(}\DataTypeTok{text =} \KeywordTok{element_text}\NormalTok{(}\DataTypeTok{size=}\DecValTok{10}\NormalTok{), }\DataTypeTok{axis.text.x =} \KeywordTok{element_text}\NormalTok{(}\DataTypeTok{angle=}\DecValTok{90}\NormalTok{, }\DataTypeTok{hjust=}\DecValTok{1}\NormalTok{)) }
\NormalTok{plot6f <-}\StringTok{ }\NormalTok{plot6f}\OperatorTok{+}\StringTok{ }\KeywordTok{labs}\NormalTok{(}\DataTypeTok{title=} \StringTok{"Sum of spending by categories June"}\NormalTok{, }\DataTypeTok{y=}\StringTok{"Spent"}\NormalTok{, }\DataTypeTok{x =} \StringTok{"Category"}\NormalTok{) }\OperatorTok{+}\KeywordTok{theme}\NormalTok{(}\DataTypeTok{text =} \KeywordTok{element_text}\NormalTok{(}\DataTypeTok{size=}\DecValTok{10}\NormalTok{), }\DataTypeTok{axis.text.x =} \KeywordTok{element_text}\NormalTok{(}\DataTypeTok{angle=}\DecValTok{90}\NormalTok{, }\DataTypeTok{hjust=}\DecValTok{1}\NormalTok{)) }


\NormalTok{plot1f}
\end{Highlighting}
\end{Shaded}

\includegraphics{BudgetRmarkdown_files/figure-latex/unnamed-chunk-3-1.pdf}

\begin{Shaded}
\begin{Highlighting}[]
\NormalTok{plot2f}
\end{Highlighting}
\end{Shaded}

\includegraphics{BudgetRmarkdown_files/figure-latex/unnamed-chunk-3-2.pdf}

\begin{Shaded}
\begin{Highlighting}[]
\NormalTok{plot3f}
\end{Highlighting}
\end{Shaded}

\includegraphics{BudgetRmarkdown_files/figure-latex/unnamed-chunk-3-3.pdf}

\begin{Shaded}
\begin{Highlighting}[]
\NormalTok{plot4f}
\end{Highlighting}
\end{Shaded}

\includegraphics{BudgetRmarkdown_files/figure-latex/unnamed-chunk-3-4.pdf}

\begin{Shaded}
\begin{Highlighting}[]
\NormalTok{plot5f}
\end{Highlighting}
\end{Shaded}

\includegraphics{BudgetRmarkdown_files/figure-latex/unnamed-chunk-3-5.pdf}

\begin{Shaded}
\begin{Highlighting}[]
\NormalTok{plot6f}
\end{Highlighting}
\end{Shaded}

\includegraphics{BudgetRmarkdown_files/figure-latex/unnamed-chunk-3-6.pdf}

\hypertarget{average-spending-for-all-months}{%
\section{Average Spending for All
Months}\label{average-spending-for-all-months}}

Now we will take the average of each category throughout all of the
months

\begin{Shaded}
\begin{Highlighting}[]
\CommentTok{# get dataframe of average by category}
\NormalTok{budget_avg <-}\StringTok{ }\NormalTok{budget_f[}\OperatorTok{-}\KeywordTok{c}\NormalTok{(}\DecValTok{2}\NormalTok{)]}
\NormalTok{budget_avg <-}\StringTok{ }\KeywordTok{data.frame}\NormalTok{(budget_avg)}
\CommentTok{# AVerage spending for all months}
\NormalTok{df_avg2 <-}\StringTok{ }\NormalTok{budget_avg }\OperatorTok
\StringTok{  }\KeywordTok{group_by}\NormalTok{(category)}\OperatorTok
\StringTok{  }\KeywordTok{summarise}\NormalTok{(}\DataTypeTok{cost =} \KeywordTok{mean}\NormalTok{(cost))}
\KeywordTok{head}\NormalTok{(df_avg2)}
\end{Highlighting}
\end{Shaded}

\begin{verbatim}
## # A tibble: 6 x 2
##   category                   cost
##   <fct>                     <dbl>
## 1 Advertising                5   
## 2 Automotive Expenses      135.  
## 3 Cable/Satellite Services  60   
## 4 Clothing/Shoes           112.  
## 5 Dues and Subscriptions     6.88
## 6 Education                 44.8
\end{verbatim}

\hypertarget{plotting-the-average-spending-for-each-cateogory-throughout-each-month}{%
\section{Plotting the average spending for each cateogory throughout
each
month}\label{plotting-the-average-spending-for-each-cateogory-throughout-each-month}}

\begin{Shaded}
\begin{Highlighting}[]
\NormalTok{plot_base3 <-}\StringTok{ }\KeywordTok{ggplot}\NormalTok{(}\DataTypeTok{data =}\NormalTok{ df_avg2, }\DataTypeTok{mapping =} \KeywordTok{aes}\NormalTok{(}\DataTypeTok{x =} \KeywordTok{reorder}\NormalTok{(category, }\OperatorTok{-}\NormalTok{cost), }\DataTypeTok{y =}\NormalTok{ cost))}

\CommentTok{# save a better-formatted version of the base plot in "plot_base_clean"}
\NormalTok{plot_base_clean3 <-}\StringTok{ }\NormalTok{plot_base3 }\OperatorTok{+}\StringTok{ }
\StringTok{  }\CommentTok{# apply basic black and white theme - this theme removes the background colour by default}
\StringTok{  }\KeywordTok{theme_bw}\NormalTok{() }\OperatorTok{+}\StringTok{ }
\StringTok{  }\CommentTok{# remove gridlines. panel.grid.major is for vertical lines, panel.grid.minor is for horizontal lines}
\StringTok{  }\KeywordTok{theme}\NormalTok{(}\DataTypeTok{panel.grid.major =} \KeywordTok{element_blank}\NormalTok{(), }\DataTypeTok{panel.grid.minor =} \KeywordTok{element_blank}\NormalTok{(),}
        \CommentTok{# remove borders}
        \DataTypeTok{panel.border =} \KeywordTok{element_blank}\NormalTok{(),}
        \CommentTok{# removing borders also removes x and y axes. Add them back}
        \DataTypeTok{axis.line =} \KeywordTok{element_line}\NormalTok{())}

\CommentTok{# PLOT AVERAGE SPENDING PER MONTH}
\NormalTok{plotf3 <-}\StringTok{ }\NormalTok{plot_base_clean3 }\OperatorTok{+}\StringTok{ }\KeywordTok{geom_bar}\NormalTok{(}\DataTypeTok{data =}\NormalTok{ df_avg2, }\DataTypeTok{stat =} \StringTok{"identity"}\NormalTok{, }\KeywordTok{aes}\NormalTok{(}\DataTypeTok{fill =}\NormalTok{ category ))  }
\NormalTok{plotf3 <-}\StringTok{ }\NormalTok{plotf3}\OperatorTok{+}\StringTok{ }\KeywordTok{labs}\NormalTok{(}\DataTypeTok{title=} \StringTok{"Sum of spending by categories Avg"}\NormalTok{, }\DataTypeTok{y=}\StringTok{"Spent"}\NormalTok{, }\DataTypeTok{x =} \StringTok{"Category"}\NormalTok{) }\OperatorTok{+}\KeywordTok{theme}\NormalTok{(}\DataTypeTok{text =} \KeywordTok{element_text}\NormalTok{(}\DataTypeTok{size=}\DecValTok{10}\NormalTok{), }\DataTypeTok{axis.text.x =} \KeywordTok{element_text}\NormalTok{(}\DataTypeTok{angle=}\DecValTok{90}\NormalTok{, }\DataTypeTok{hjust=}\DecValTok{1}\NormalTok{)) }
\NormalTok{plotf3}
\end{Highlighting}
\end{Shaded}

\includegraphics{BudgetRmarkdown_files/figure-latex/unnamed-chunk-5-1.pdf}

\hypertarget{get-the-difference-between-the-overall-average-for-each-category-and-the-spending-for-that-particular-month}{%
\section{Get the difference between the overall average for each
category and the spending for that particular
month}\label{get-the-difference-between-the-overall-average-for-each-category-and-the-spending-for-that-particular-month}}

AVERAGE COST MINUS WHAT WE SPENT if it is positive we spent less than
what we needed to if it is negative we spent more than we wanted to

\begin{Shaded}
\begin{Highlighting}[]
\CommentTok{# GET DIFFERENCE OF AVERAGE AND REGULAR}
\NormalTok{month1 <-}\StringTok{ }\NormalTok{budget_f[budget_f}\OperatorTok{$}\NormalTok{month }\OperatorTok{==}\DecValTok{1}\NormalTok{,]; month1 <-}\StringTok{ }\NormalTok{month1[}\OperatorTok{-}\KeywordTok{c}\NormalTok{(}\DecValTok{2}\NormalTok{)];}
\NormalTok{month2 <-}\StringTok{ }\NormalTok{budget_f[budget_f}\OperatorTok{$}\NormalTok{month }\OperatorTok{==}\DecValTok{2}\NormalTok{,]; month2 <-}\StringTok{ }\NormalTok{month2[}\OperatorTok{-}\KeywordTok{c}\NormalTok{(}\DecValTok{2}\NormalTok{)];}
\NormalTok{month3 <-}\StringTok{ }\NormalTok{budget_f[budget_f}\OperatorTok{$}\NormalTok{month }\OperatorTok{==}\DecValTok{3}\NormalTok{,]; month3 <-}\StringTok{ }\NormalTok{month3[}\OperatorTok{-}\KeywordTok{c}\NormalTok{(}\DecValTok{2}\NormalTok{)];}
\NormalTok{month4 <-}\StringTok{ }\NormalTok{budget_f[budget_f}\OperatorTok{$}\NormalTok{month }\OperatorTok{==}\DecValTok{4}\NormalTok{,]; month4 <-}\StringTok{ }\NormalTok{month4[}\OperatorTok{-}\KeywordTok{c}\NormalTok{(}\DecValTok{2}\NormalTok{)];}
\NormalTok{month5 <-}\StringTok{ }\NormalTok{budget_f[budget_f}\OperatorTok{$}\NormalTok{month }\OperatorTok{==}\DecValTok{5}\NormalTok{,]; month5 <-}\StringTok{ }\NormalTok{month5[}\OperatorTok{-}\KeywordTok{c}\NormalTok{(}\DecValTok{2}\NormalTok{)];}
\NormalTok{month6 <-}\StringTok{ }\NormalTok{budget_f[budget_f}\OperatorTok{$}\NormalTok{month }\OperatorTok{==}\DecValTok{6}\NormalTok{,]; month6 <-}\StringTok{ }\NormalTok{month6[}\OperatorTok{-}\KeywordTok{c}\NormalTok{(}\DecValTok{2}\NormalTok{)];}

\KeywordTok{colnames}\NormalTok{(df_avg2)[}\DecValTok{2}\NormalTok{] <-}\StringTok{ "Avg cost"}

\NormalTok{month1f <-}\StringTok{ }\KeywordTok{merge}\NormalTok{(month1,df_avg2, }\DataTypeTok{by =} \KeywordTok{c}\NormalTok{(}\StringTok{"category"}\NormalTok{),}\DataTypeTok{all.x=} \OtherTok{TRUE}\NormalTok{)}\CommentTok{#; month1f <- month1f[-c(2,3)];}
\NormalTok{month2f <-}\StringTok{ }\KeywordTok{merge}\NormalTok{(month2,df_avg2, }\DataTypeTok{by =} \KeywordTok{c}\NormalTok{(}\StringTok{"category"}\NormalTok{),}\DataTypeTok{all.x=} \OtherTok{TRUE}\NormalTok{)}\CommentTok{#; month2f <- month2f[-c(2,3)];}
\NormalTok{month3f <-}\StringTok{ }\KeywordTok{merge}\NormalTok{(month3,df_avg2, }\DataTypeTok{by =} \KeywordTok{c}\NormalTok{(}\StringTok{"category"}\NormalTok{),}\DataTypeTok{all.x=} \OtherTok{TRUE}\NormalTok{)}\CommentTok{#; month3f <- month3f[-c(2,3)];}
\NormalTok{month4f <-}\StringTok{ }\KeywordTok{merge}\NormalTok{(month4,df_avg2, }\DataTypeTok{by =} \KeywordTok{c}\NormalTok{(}\StringTok{"category"}\NormalTok{),}\DataTypeTok{all.x=} \OtherTok{TRUE}\NormalTok{)}\CommentTok{#; month4f <- month4f[-c(2,3)];}
\NormalTok{month5f <-}\StringTok{ }\KeywordTok{merge}\NormalTok{(month5,df_avg2, }\DataTypeTok{by =} \KeywordTok{c}\NormalTok{(}\StringTok{"category"}\NormalTok{),}\DataTypeTok{all.x=} \OtherTok{TRUE}\NormalTok{)}\CommentTok{#; month5f <- month5f[-c(2,3)];}
\NormalTok{month6f <-}\StringTok{ }\KeywordTok{merge}\NormalTok{(month6,df_avg2, }\DataTypeTok{by =} \KeywordTok{c}\NormalTok{(}\StringTok{"category"}\NormalTok{),}\DataTypeTok{all.x=} \OtherTok{TRUE}\NormalTok{)}\CommentTok{#; month6f <- month6f[-c(2,3)];}

\CommentTok{# AVERAGE COST MINUS WHAT WE SPENT }
\CommentTok{# iff it is positive we spent less than what we needed to if}
\CommentTok{# if it is negative we spent more than we wanted to }
\NormalTok{month1f}\OperatorTok{$}\NormalTok{cost2 <-}\StringTok{ }\NormalTok{month1f}\OperatorTok{$}\StringTok{`}\DataTypeTok{Avg cost}\StringTok{`} \OperatorTok{-}\StringTok{ }\NormalTok{month1f}\OperatorTok{$}\NormalTok{cost}\CommentTok{#; month1f <- month1f[-c(2,3)];}
\NormalTok{month2f}\OperatorTok{$}\NormalTok{cost2 <-}\StringTok{ }\NormalTok{month2f}\OperatorTok{$}\StringTok{`}\DataTypeTok{Avg cost}\StringTok{`} \OperatorTok{-}\StringTok{ }\NormalTok{month2f}\OperatorTok{$}\NormalTok{cost}\CommentTok{#;}
\NormalTok{month3f}\OperatorTok{$}\NormalTok{cost2 <-}\StringTok{ }\NormalTok{month3f}\OperatorTok{$}\StringTok{`}\DataTypeTok{Avg cost}\StringTok{`} \OperatorTok{-}\StringTok{ }\NormalTok{month3f}\OperatorTok{$}\NormalTok{cost}\CommentTok{#;}
\NormalTok{month4f}\OperatorTok{$}\NormalTok{cost2 <-}\StringTok{ }\NormalTok{month4f}\OperatorTok{$}\StringTok{`}\DataTypeTok{Avg cost}\StringTok{`} \OperatorTok{-}\StringTok{ }\NormalTok{month4f}\OperatorTok{$}\NormalTok{cost}\CommentTok{#;}
\NormalTok{month5f}\OperatorTok{$}\NormalTok{cost2 <-}\StringTok{ }\NormalTok{month5f}\OperatorTok{$}\StringTok{`}\DataTypeTok{Avg cost}\StringTok{`} \OperatorTok{-}\StringTok{ }\NormalTok{month5f}\OperatorTok{$}\NormalTok{cost}\CommentTok{#;}
\NormalTok{month6f}\OperatorTok{$}\NormalTok{cost2 <-}\StringTok{ }\NormalTok{month6f}\OperatorTok{$}\StringTok{`}\DataTypeTok{Avg cost}\StringTok{`} \OperatorTok{-}\StringTok{ }\NormalTok{month6f}\OperatorTok{$}\NormalTok{cost}\CommentTok{#;}

\CommentTok{# Remove irrelevant columns}
\NormalTok{ month1f <-}\StringTok{ }\NormalTok{month1f[}\OperatorTok{-}\KeywordTok{c}\NormalTok{(}\DecValTok{2}\NormalTok{,}\DecValTok{3}\NormalTok{)];}
\NormalTok{ month2f <-}\StringTok{ }\NormalTok{month2f[}\OperatorTok{-}\KeywordTok{c}\NormalTok{(}\DecValTok{2}\NormalTok{,}\DecValTok{3}\NormalTok{)];}
\NormalTok{ month3f <-}\StringTok{ }\NormalTok{month3f[}\OperatorTok{-}\KeywordTok{c}\NormalTok{(}\DecValTok{2}\NormalTok{,}\DecValTok{3}\NormalTok{)];}
\NormalTok{ month4f <-}\StringTok{ }\NormalTok{month4f[}\OperatorTok{-}\KeywordTok{c}\NormalTok{(}\DecValTok{2}\NormalTok{,}\DecValTok{3}\NormalTok{)];}
\NormalTok{ month5f <-}\StringTok{ }\NormalTok{month5f[}\OperatorTok{-}\KeywordTok{c}\NormalTok{(}\DecValTok{2}\NormalTok{,}\DecValTok{3}\NormalTok{)];}
\NormalTok{ month6f <-}\StringTok{ }\NormalTok{month6f[}\OperatorTok{-}\KeywordTok{c}\NormalTok{(}\DecValTok{2}\NormalTok{,}\DecValTok{3}\NormalTok{)];}
 
\NormalTok{ plot_base1 <-}\StringTok{ }\KeywordTok{ggplot}\NormalTok{(}\DataTypeTok{data =}\NormalTok{ month1f, }\DataTypeTok{mapping =} \KeywordTok{aes}\NormalTok{(}\DataTypeTok{x =} \KeywordTok{reorder}\NormalTok{(category, }\OperatorTok{-}\NormalTok{cost2), }\DataTypeTok{y =}\NormalTok{ cost2))}
\NormalTok{ plot_base2 <-}\StringTok{ }\KeywordTok{ggplot}\NormalTok{(}\DataTypeTok{data =}\NormalTok{ month2f, }\DataTypeTok{mapping =} \KeywordTok{aes}\NormalTok{(}\DataTypeTok{x =} \KeywordTok{reorder}\NormalTok{(category, }\OperatorTok{-}\NormalTok{cost2), }\DataTypeTok{y =}\NormalTok{ cost2))}
\NormalTok{ plot_base3 <-}\StringTok{ }\KeywordTok{ggplot}\NormalTok{(}\DataTypeTok{data =}\NormalTok{ month3f, }\DataTypeTok{mapping =} \KeywordTok{aes}\NormalTok{(}\DataTypeTok{x =} \KeywordTok{reorder}\NormalTok{(category, }\OperatorTok{-}\NormalTok{cost2), }\DataTypeTok{y =}\NormalTok{ cost2))}
\NormalTok{ plot_base4 <-}\StringTok{ }\KeywordTok{ggplot}\NormalTok{(}\DataTypeTok{data =}\NormalTok{ month4f, }\DataTypeTok{mapping =} \KeywordTok{aes}\NormalTok{(}\DataTypeTok{x =} \KeywordTok{reorder}\NormalTok{(category, }\OperatorTok{-}\NormalTok{cost2), }\DataTypeTok{y =}\NormalTok{ cost2))}
\NormalTok{ plot_base5 <-}\StringTok{ }\KeywordTok{ggplot}\NormalTok{(}\DataTypeTok{data =}\NormalTok{ month5f, }\DataTypeTok{mapping =} \KeywordTok{aes}\NormalTok{(}\DataTypeTok{x =} \KeywordTok{reorder}\NormalTok{(category, }\OperatorTok{-}\NormalTok{cost2), }\DataTypeTok{y =}\NormalTok{ cost2))}
\NormalTok{ plot_base6 <-}\StringTok{ }\KeywordTok{ggplot}\NormalTok{(}\DataTypeTok{data =}\NormalTok{ month6f, }\DataTypeTok{mapping =} \KeywordTok{aes}\NormalTok{(}\DataTypeTok{x =} \KeywordTok{reorder}\NormalTok{(category, }\OperatorTok{-}\NormalTok{cost2), }\DataTypeTok{y =}\NormalTok{ cost2))}

 
  \CommentTok{# save a better-formatted version of the base plot in "plot_base_clean"}
\NormalTok{ plot_base_clean1 <-}\StringTok{ }\NormalTok{plot_base1 }\OperatorTok{+}\StringTok{ }
\StringTok{   }\CommentTok{# apply basic black and white theme - this theme removes the background colour by default}
\StringTok{   }\KeywordTok{theme_bw}\NormalTok{() }\OperatorTok{+}\StringTok{ }
\StringTok{   }\CommentTok{# remove gridlines. panel.grid.major is for vertical lines, panel.grid.minor is for horizontal lines}
\StringTok{   }\KeywordTok{theme}\NormalTok{(}\DataTypeTok{panel.grid.major =} \KeywordTok{element_blank}\NormalTok{(), }\DataTypeTok{panel.grid.minor =} \KeywordTok{element_blank}\NormalTok{(),}
         \CommentTok{# remove borders}
         \DataTypeTok{panel.border =} \KeywordTok{element_blank}\NormalTok{(),}
         \CommentTok{# removing borders also removes x and y axes. Add them back}
         \DataTypeTok{axis.line =} \KeywordTok{element_line}\NormalTok{())}
 
 \CommentTok{# save a better-formatted version of the base plot in "plot_base_clean"}
\NormalTok{ plot_base_clean2 <-}\StringTok{ }\NormalTok{plot_base2 }\OperatorTok{+}\StringTok{ }
\StringTok{   }\CommentTok{# apply basic black and white theme - this theme removes the background colour by default}
\StringTok{   }\KeywordTok{theme_bw}\NormalTok{() }\OperatorTok{+}\StringTok{ }
\StringTok{   }\CommentTok{# remove gridlines. panel.grid.major is for vertical lines, panel.grid.minor is for horizontal lines}
\StringTok{   }\KeywordTok{theme}\NormalTok{(}\DataTypeTok{panel.grid.major =} \KeywordTok{element_blank}\NormalTok{(), }\DataTypeTok{panel.grid.minor =} \KeywordTok{element_blank}\NormalTok{(),}
         \CommentTok{# remove borders}
         \DataTypeTok{panel.border =} \KeywordTok{element_blank}\NormalTok{(),}
         \CommentTok{# removing borders also removes x and y axes. Add them back}
         \DataTypeTok{axis.line =} \KeywordTok{element_line}\NormalTok{())}
 
 \CommentTok{# save a better-formatted version of the base plot in "plot_base_clean"}
\NormalTok{ plot_base_clean3 <-}\StringTok{ }\NormalTok{plot_base3 }\OperatorTok{+}\StringTok{ }
\StringTok{   }\CommentTok{# apply basic black and white theme - this theme removes the background colour by default}
\StringTok{   }\KeywordTok{theme_bw}\NormalTok{() }\OperatorTok{+}\StringTok{ }
\StringTok{   }\CommentTok{# remove gridlines. panel.grid.major is for vertical lines, panel.grid.minor is for horizontal lines}
\StringTok{   }\KeywordTok{theme}\NormalTok{(}\DataTypeTok{panel.grid.major =} \KeywordTok{element_blank}\NormalTok{(), }\DataTypeTok{panel.grid.minor =} \KeywordTok{element_blank}\NormalTok{(),}
         \CommentTok{# remove borders}
         \DataTypeTok{panel.border =} \KeywordTok{element_blank}\NormalTok{(),}
         \CommentTok{# removing borders also removes x and y axes. Add them back}
         \DataTypeTok{axis.line =} \KeywordTok{element_line}\NormalTok{())}

 \CommentTok{# save a better-formatted version of the base plot in "plot_base_clean"}
\NormalTok{ plot_base_clean4 <-}\StringTok{ }\NormalTok{plot_base4 }\OperatorTok{+}\StringTok{ }
\StringTok{   }\CommentTok{# apply basic black and white theme - this theme removes the background colour by default}
\StringTok{   }\KeywordTok{theme_bw}\NormalTok{() }\OperatorTok{+}\StringTok{ }
\StringTok{   }\CommentTok{# remove gridlines. panel.grid.major is for vertical lines, panel.grid.minor is for horizontal lines}
\StringTok{   }\KeywordTok{theme}\NormalTok{(}\DataTypeTok{panel.grid.major =} \KeywordTok{element_blank}\NormalTok{(), }\DataTypeTok{panel.grid.minor =} \KeywordTok{element_blank}\NormalTok{(),}
         \CommentTok{# remove borders}
         \DataTypeTok{panel.border =} \KeywordTok{element_blank}\NormalTok{(),}
         \CommentTok{# removing borders also removes x and y axes. Add them back}
         \DataTypeTok{axis.line =} \KeywordTok{element_line}\NormalTok{())}
 

 \CommentTok{# save a better-formatted version of the base plot in "plot_base_clean"}
\NormalTok{ plot_base_clean5 <-}\StringTok{ }\NormalTok{plot_base5 }\OperatorTok{+}\StringTok{ }
\StringTok{   }\CommentTok{# apply basic black and white theme - this theme removes the background colour by default}
\StringTok{   }\KeywordTok{theme_bw}\NormalTok{() }\OperatorTok{+}\StringTok{ }
\StringTok{   }\CommentTok{# remove gridlines. panel.grid.major is for vertical lines, panel.grid.minor is for horizontal lines}
\StringTok{   }\KeywordTok{theme}\NormalTok{(}\DataTypeTok{panel.grid.major =} \KeywordTok{element_blank}\NormalTok{(), }\DataTypeTok{panel.grid.minor =} \KeywordTok{element_blank}\NormalTok{(),}
         \CommentTok{# remove borders}
         \DataTypeTok{panel.border =} \KeywordTok{element_blank}\NormalTok{(),}
         \CommentTok{# removing borders also removes x and y axes. Add them back}
         \DataTypeTok{axis.line =} \KeywordTok{element_line}\NormalTok{())}

 
 \CommentTok{# save a better-formatted version of the base plot in "plot_base_clean"}
\NormalTok{ plot_base_clean6 <-}\StringTok{ }\NormalTok{plot_base6 }\OperatorTok{+}\StringTok{ }
\StringTok{   }\CommentTok{# apply basic black and white theme - this theme removes the background colour by default}
\StringTok{   }\KeywordTok{theme_bw}\NormalTok{() }\OperatorTok{+}\StringTok{ }
\StringTok{   }\CommentTok{# remove gridlines. panel.grid.major is for vertical lines, panel.grid.minor is for horizontal lines}
\StringTok{   }\KeywordTok{theme}\NormalTok{(}\DataTypeTok{panel.grid.major =} \KeywordTok{element_blank}\NormalTok{(), }\DataTypeTok{panel.grid.minor =} \KeywordTok{element_blank}\NormalTok{(),}
         \CommentTok{# remove borders}
         \DataTypeTok{panel.border =} \KeywordTok{element_blank}\NormalTok{(),}
         \CommentTok{# removing borders also removes x and y axes. Add them back}
         \DataTypeTok{axis.line =} \KeywordTok{element_line}\NormalTok{())}

  \CommentTok{# PLOT AVERAGE SPENDING PER MONTH}
\NormalTok{ plotf1 <-}\StringTok{ }\NormalTok{plot_base_clean1 }\OperatorTok{+}\StringTok{ }\KeywordTok{geom_bar}\NormalTok{(}\DataTypeTok{data =}\NormalTok{ month1f, }\DataTypeTok{stat =} \StringTok{"identity"}\NormalTok{, }\KeywordTok{aes}\NormalTok{(}\DataTypeTok{fill =}\NormalTok{ category))  }
\NormalTok{ plotf1 <-}\StringTok{ }\NormalTok{plotf1}\OperatorTok{+}\StringTok{ }\KeywordTok{labs}\NormalTok{(}\DataTypeTok{title=} \StringTok{"Sum of spending by categories Avg January"}\NormalTok{, }\DataTypeTok{y=}\StringTok{"Spent"}\NormalTok{, }\DataTypeTok{x =} \StringTok{"Category"}\NormalTok{) }\OperatorTok{+}\KeywordTok{theme}\NormalTok{(}\DataTypeTok{text =} \KeywordTok{element_text}\NormalTok{(}\DataTypeTok{size=}\DecValTok{10}\NormalTok{), }\DataTypeTok{axis.text.x =} \KeywordTok{element_text}\NormalTok{(}\DataTypeTok{angle=}\DecValTok{90}\NormalTok{, }\DataTypeTok{hjust=}\DecValTok{1}\NormalTok{)) }
\NormalTok{ plotf1}
\end{Highlighting}
\end{Shaded}

\includegraphics{BudgetRmarkdown_files/figure-latex/unnamed-chunk-6-1.pdf}

\begin{Shaded}
\begin{Highlighting}[]
  \CommentTok{# PLOT AVERAGE SPENDING PER MONTH}
\NormalTok{ plotf2 <-}\StringTok{ }\NormalTok{plot_base_clean2 }\OperatorTok{+}\StringTok{ }\KeywordTok{geom_bar}\NormalTok{(}\DataTypeTok{data =}\NormalTok{ month2f, }\DataTypeTok{stat =} \StringTok{"identity"}\NormalTok{, }\KeywordTok{aes}\NormalTok{(}\DataTypeTok{fill =}\NormalTok{ category ))  }
\NormalTok{ plotf2 <-}\StringTok{ }\NormalTok{plotf2}\OperatorTok{+}\StringTok{ }\KeywordTok{labs}\NormalTok{(}\DataTypeTok{title=} \StringTok{"Sum of spending by categories Avg Febuary"}\NormalTok{, }\DataTypeTok{y=}\StringTok{"Spent"}\NormalTok{, }\DataTypeTok{x =} \StringTok{"Category"}\NormalTok{) }\OperatorTok{+}\KeywordTok{theme}\NormalTok{(}\DataTypeTok{text =} \KeywordTok{element_text}\NormalTok{(}\DataTypeTok{size=}\DecValTok{10}\NormalTok{), }\DataTypeTok{axis.text.x =} \KeywordTok{element_text}\NormalTok{(}\DataTypeTok{angle=}\DecValTok{90}\NormalTok{, }\DataTypeTok{hjust=}\DecValTok{1}\NormalTok{)) }
\NormalTok{ plotf2}
\end{Highlighting}
\end{Shaded}

\includegraphics{BudgetRmarkdown_files/figure-latex/unnamed-chunk-6-2.pdf}

\begin{Shaded}
\begin{Highlighting}[]
 \CommentTok{# PLOT AVERAGE SPENDING PER MONTH}
\NormalTok{ plotf3 <-}\StringTok{ }\NormalTok{plot_base_clean3 }\OperatorTok{+}\StringTok{ }\KeywordTok{geom_bar}\NormalTok{(}\DataTypeTok{data =}\NormalTok{ month3f, }\DataTypeTok{stat =} \StringTok{"identity"}\NormalTok{, }\KeywordTok{aes}\NormalTok{(}\DataTypeTok{fill =}\NormalTok{ category ))  }
\NormalTok{ plotf3 <-}\StringTok{ }\NormalTok{plotf3}\OperatorTok{+}\StringTok{ }\KeywordTok{labs}\NormalTok{(}\DataTypeTok{title=} \StringTok{"Sum of spending by categories Avg March"}\NormalTok{, }\DataTypeTok{y=}\StringTok{"Spent"}\NormalTok{, }\DataTypeTok{x =} \StringTok{"Category"}\NormalTok{) }\OperatorTok{+}\KeywordTok{theme}\NormalTok{(}\DataTypeTok{text =} \KeywordTok{element_text}\NormalTok{(}\DataTypeTok{size=}\DecValTok{10}\NormalTok{), }\DataTypeTok{axis.text.x =} \KeywordTok{element_text}\NormalTok{(}\DataTypeTok{angle=}\DecValTok{90}\NormalTok{, }\DataTypeTok{hjust=}\DecValTok{1}\NormalTok{)) }
\NormalTok{ plotf3}
\end{Highlighting}
\end{Shaded}

\includegraphics{BudgetRmarkdown_files/figure-latex/unnamed-chunk-6-3.pdf}

\begin{Shaded}
\begin{Highlighting}[]
 \CommentTok{# PLOT AVERAGE SPENDING PER MONTH}
\NormalTok{ plotf4 <-}\StringTok{ }\NormalTok{plot_base_clean4 }\OperatorTok{+}\StringTok{ }\KeywordTok{geom_bar}\NormalTok{(}\DataTypeTok{data =}\NormalTok{ month4f, }\DataTypeTok{stat =} \StringTok{"identity"}\NormalTok{, }\KeywordTok{aes}\NormalTok{(}\DataTypeTok{fill =}\NormalTok{ category ))  }
\NormalTok{ plotf4 <-}\StringTok{ }\NormalTok{plotf4}\OperatorTok{+}\StringTok{ }\KeywordTok{labs}\NormalTok{(}\DataTypeTok{title=} \StringTok{"Sum of spending by categories Avg April"}\NormalTok{, }\DataTypeTok{y=}\StringTok{"Spent"}\NormalTok{, }\DataTypeTok{x =} \StringTok{"Category"}\NormalTok{) }\OperatorTok{+}\KeywordTok{theme}\NormalTok{(}\DataTypeTok{text =} \KeywordTok{element_text}\NormalTok{(}\DataTypeTok{size=}\DecValTok{10}\NormalTok{), }\DataTypeTok{axis.text.x =} \KeywordTok{element_text}\NormalTok{(}\DataTypeTok{angle=}\DecValTok{90}\NormalTok{, }\DataTypeTok{hjust=}\DecValTok{1}\NormalTok{)) }
\NormalTok{ plotf4}
\end{Highlighting}
\end{Shaded}

\includegraphics{BudgetRmarkdown_files/figure-latex/unnamed-chunk-6-4.pdf}

\begin{Shaded}
\begin{Highlighting}[]
 \CommentTok{# PLOT AVERAGE SPENDING PER MONTH}
\NormalTok{ plotf5 <-}\StringTok{ }\NormalTok{plot_base_clean5 }\OperatorTok{+}\StringTok{ }\KeywordTok{geom_bar}\NormalTok{(}\DataTypeTok{data =}\NormalTok{ month5f, }\DataTypeTok{stat =} \StringTok{"identity"}\NormalTok{, }\KeywordTok{aes}\NormalTok{(}\DataTypeTok{fill =}\NormalTok{ category ))  }
\NormalTok{ plotf5 <-}\StringTok{ }\NormalTok{plotf5}\OperatorTok{+}\StringTok{ }\KeywordTok{labs}\NormalTok{(}\DataTypeTok{title=} \StringTok{"Sum of spending by categories Avg May"}\NormalTok{, }\DataTypeTok{y=}\StringTok{"Spent"}\NormalTok{, }\DataTypeTok{x =} \StringTok{"Category"}\NormalTok{) }\OperatorTok{+}\KeywordTok{theme}\NormalTok{(}\DataTypeTok{text =} \KeywordTok{element_text}\NormalTok{(}\DataTypeTok{size=}\DecValTok{10}\NormalTok{), }\DataTypeTok{axis.text.x =} \KeywordTok{element_text}\NormalTok{(}\DataTypeTok{angle=}\DecValTok{90}\NormalTok{, }\DataTypeTok{hjust=}\DecValTok{1}\NormalTok{)) }
\NormalTok{ plotf5}
\end{Highlighting}
\end{Shaded}

\includegraphics{BudgetRmarkdown_files/figure-latex/unnamed-chunk-6-5.pdf}

\begin{Shaded}
\begin{Highlighting}[]
 \CommentTok{# PLOT AVERAGE SPENDING PER MONTH}
\NormalTok{ plotf6 <-}\StringTok{ }\NormalTok{plot_base_clean6 }\OperatorTok{+}\StringTok{ }\KeywordTok{geom_bar}\NormalTok{(}\DataTypeTok{data =}\NormalTok{ month6f, }\DataTypeTok{stat =} \StringTok{"identity"}\NormalTok{, }\KeywordTok{aes}\NormalTok{(}\DataTypeTok{fill =}\NormalTok{ category ))  }
\NormalTok{ plotf6 <-}\StringTok{ }\NormalTok{plotf6}\OperatorTok{+}\StringTok{ }\KeywordTok{labs}\NormalTok{(}\DataTypeTok{title=} \StringTok{"Sum of spending by categories Avg June"}\NormalTok{, }\DataTypeTok{y=}\StringTok{"Spent"}\NormalTok{, }\DataTypeTok{x =} \StringTok{"Category"}\NormalTok{) }\OperatorTok{+}\KeywordTok{theme}\NormalTok{(}\DataTypeTok{text =} \KeywordTok{element_text}\NormalTok{(}\DataTypeTok{size=}\DecValTok{10}\NormalTok{), }\DataTypeTok{axis.text.x =} \KeywordTok{element_text}\NormalTok{(}\DataTypeTok{angle=}\DecValTok{90}\NormalTok{, }\DataTypeTok{hjust=}\DecValTok{1}\NormalTok{)) }
\NormalTok{ plotf6}
\end{Highlighting}
\end{Shaded}

\includegraphics{BudgetRmarkdown_files/figure-latex/unnamed-chunk-6-6.pdf}

\hypertarget{now-get-average-spending-for-the-total-credit-card-amount}{%
\section{Now get average spending for the total credit card
amount}\label{now-get-average-spending-for-the-total-credit-card-amount}}

\begin{Shaded}
\begin{Highlighting}[]
\NormalTok{budget_f2 <-}\StringTok{ }\KeywordTok{aggregate}\NormalTok{(budget[,}\KeywordTok{c}\NormalTok{(}\DecValTok{4}\NormalTok{)], }\DataTypeTok{by =} \KeywordTok{list}\NormalTok{(budget}\OperatorTok{$}\NormalTok{category,budget}\OperatorTok{$}\NormalTok{month), }\DataTypeTok{FUN=}\NormalTok{ sum)}
\KeywordTok{colnames}\NormalTok{(budget_f2)[}\DecValTok{1}\NormalTok{] <-}\StringTok{ "category"}
\KeywordTok{colnames}\NormalTok{(budget_f2)[}\DecValTok{2}\NormalTok{] <-}\StringTok{ "month"}
\KeywordTok{colnames}\NormalTok{(budget_f2)[}\DecValTok{3}\NormalTok{] <-}\StringTok{ "cost"}
\NormalTok{budget_f2 <-}\StringTok{ }\NormalTok{budget_f2[}\KeywordTok{order}\NormalTok{(budget_f2}\OperatorTok{$}\NormalTok{month, budget_f2}\OperatorTok{$}\NormalTok{cost),]}
\NormalTok{budget_f2}\OperatorTok{$}\NormalTok{cost <-}\StringTok{ }\KeywordTok{as.numeric}\NormalTok{(budget_f2}\OperatorTok{$}\NormalTok{cost)}
\NormalTok{budget_f2 <-}\StringTok{ }\KeywordTok{as.data.frame}\NormalTok{(budget_f2)}

\CommentTok{# AVerage spending for all months}
\NormalTok{budget_f3 <-}\StringTok{ }\NormalTok{budget_f2 }\OperatorTok
\StringTok{  }\KeywordTok{group_by}\NormalTok{(category)}\OperatorTok
\StringTok{  }\KeywordTok{summarise}\NormalTok{(}\DataTypeTok{cost =} \KeywordTok{mean}\NormalTok{(cost))}
\KeywordTok{head}\NormalTok{(budget_f3)}
\end{Highlighting}
\end{Shaded}

\begin{verbatim}
## # A tibble: 6 x 2
##   category                    cost
##   <fct>                      <dbl>
## 1 Advertising                 5   
## 2 Automotive Expenses       135.  
## 3 Cable/Satellite Services   60   
## 4 Clothing/Shoes            112.  
## 5 Credit Card Payments     2684.  
## 6 Dues and Subscriptions      6.88
\end{verbatim}

\begin{Shaded}
\begin{Highlighting}[]
\NormalTok{plot_base4f <-}\StringTok{ }\KeywordTok{ggplot}\NormalTok{(}\DataTypeTok{data =}\NormalTok{ budget_f3, }\DataTypeTok{mapping =} \KeywordTok{aes}\NormalTok{(}\DataTypeTok{x =} \KeywordTok{reorder}\NormalTok{(category, }\OperatorTok{-}\NormalTok{cost), }\DataTypeTok{y =}\NormalTok{ cost))}

\CommentTok{# save a better-formatted version of the base plot in "plot_base_clean"}
\NormalTok{plot_base_clean3 <-}\StringTok{ }\NormalTok{plot_base4f }\OperatorTok{+}\StringTok{ }
\StringTok{  }\CommentTok{# apply basic black and white theme - this theme removes the background colour by default}
\StringTok{  }\KeywordTok{theme_bw}\NormalTok{() }\OperatorTok{+}\StringTok{ }
\StringTok{  }\CommentTok{# remove gridlines. panel.grid.major is for vertical lines, panel.grid.minor is for horizontal lines}
\StringTok{  }\KeywordTok{theme}\NormalTok{(}\DataTypeTok{panel.grid.major =} \KeywordTok{element_blank}\NormalTok{(), }\DataTypeTok{panel.grid.minor =} \KeywordTok{element_blank}\NormalTok{(),}
        \CommentTok{# remove borders}
        \DataTypeTok{panel.border =} \KeywordTok{element_blank}\NormalTok{(),}
        \CommentTok{# removing borders also removes x and y axes. Add them back}
        \DataTypeTok{axis.line =} \KeywordTok{element_line}\NormalTok{())}

\CommentTok{# PLOT AVERAGE SPENDING PER MONTH}
\NormalTok{plotf3f <-}\StringTok{ }\NormalTok{plot_base_clean3 }\OperatorTok{+}\StringTok{ }\KeywordTok{geom_bar}\NormalTok{(}\DataTypeTok{data =}\NormalTok{ budget_f3, }\DataTypeTok{stat =} \StringTok{"identity"}\NormalTok{, }\KeywordTok{aes}\NormalTok{(}\DataTypeTok{fill =}\NormalTok{ category))  }
\NormalTok{plotf3f <-}\StringTok{ }\NormalTok{plotf3f}\OperatorTok{+}\StringTok{ }\KeywordTok{labs}\NormalTok{(}\DataTypeTok{title=} \StringTok{"Sum of spending by categories Avg including credit card"}\NormalTok{, }\DataTypeTok{y=}\StringTok{"Spent"}\NormalTok{, }\DataTypeTok{x =} \StringTok{"Category"}\NormalTok{) }\OperatorTok{+}\KeywordTok{theme}\NormalTok{(}\DataTypeTok{text =} \KeywordTok{element_text}\NormalTok{(}\DataTypeTok{size=}\DecValTok{10}\NormalTok{), }\DataTypeTok{axis.text.x =} \KeywordTok{element_text}\NormalTok{(}\DataTypeTok{angle=}\DecValTok{90}\NormalTok{, }\DataTypeTok{hjust=}\DecValTok{1}\NormalTok{)) }
\NormalTok{plotf3f}
\end{Highlighting}
\end{Shaded}

\includegraphics{BudgetRmarkdown_files/figure-latex/unnamed-chunk-7-1.pdf}

\end{document}
